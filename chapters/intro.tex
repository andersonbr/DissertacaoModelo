\chapter{Introdução}
\label{ch:introducao}
A análise de redes sociais tem sido utilizada para a investigação de problemas
tão diversos quanto a difusão de inovações \citep{Coleman1966}, oportunidades de
emprego \citep{Granovetter1995}, prevenção contra fraude \citep{Neville2005} e
marketing \citep{Domingos2001}. Muito dessa pesquisa inicial se baseia em redes
pequenas em torno de indivíduos escolhidos por amostragem
\citep{Wasserman}\citep{Newman2006}, porém a recente disponibilidade de
informações sobre as conexões entre indíviduos através de sites de
relacionamentos na internet permitiu o desenvolvimento da análise de redes em
larga escala \citep{Boyd2007}. Não obstante, ainda há carência de modelos
dinâmicos para representar redes sociais observadas a partir do fenômeno
digital \citep{Xiang2010} e é justamente nesse ponto que o trabalho atual se
concentra. Nosso objetivo é avaliar as dificuldades, parâmetros e modelos
existentes para a representação e modelagem não-supervisionada de redes sociais
digitais em larga escala e tempo real, especificamente para aplicações que façam
uso da rede para identificar atores chaves em processos de difusão de
conhecimento, inovações e recursos.

\section{Problema}
\label{sec:problema}

Em 1996, Sabeer Bhatia e Jack Smith fundam um serviço de e-mail baseado
puramente em HTML chamado de Hotmail. Um ano e meio depois, o Hotmail já contava
com uma base de 12 milhões de usuários, sendo vendido para a Microsoft por US\$
400 milhões. Sua concorrente mais próxima, o serviço de e-mail Juno, levara o
dobro do tempo para conseguir um terço dessa quantidade de usuários, 4 milhões,
gastando em torno de US\$ 20 milhões em publicidade. O Hotmail utilizou menos de
US\$ 500 mil em propaganda \citep{Jurvetson1997}. Esse é o caso clássico de
\emph{marketing} viral.

O Hotmail e muitos outros serviços na internet depois dele utilizam-se do boca a
boca de seus usuários para se promover. O potencial de \emph{marketing} de um
indivíduo depende do seu entorno na rede social, isto é, a quem ele está
conectado através de interações e de sua capacidade de influenciá-los. Aqueles
com grande potencial são os chamados atores-chaves e através da análise de
influência é possível mapeá-los. Uma vez identificados, os atores-chaves podem
ser envolvidos em diferentes processos de difusão pela rede atuando como fontes
de informação e provocando um efeito cascata em maior escala que aquele
provocado pela escolha aleatória das fontes. O problema em questão é como
modelar a rede, quais dados levar em consideração, quais ferramentas usar para a
análise e quão confiável será os resultados.

\section{Objetivos}
\label{sec:objetivos}

A pergunta central que este trabalho busca responder é: ``como fazer a análise
de influência em redes sociais digitais?``. Para tal estabelecemos os seguintes
objetivos específicos:

\begin{itemize}
  \item definir redes sociais digitais;
  \item identificar uma metodologia apropriada para coletar, medir, armazenar os
  dados;
  \item apresentar ferramentas para a análise de influência, com suas vantagens
  e desvantagens;
  \item apontar caminhos futuros.
\end{itemize}

\section{Redes sociais}
\label{ch:redes}
O estudo de rede sociais inicia nas décadas de 40 e 50, inicialmente voltado para
o estudo de pequenos grupos de individuos e suas interações, a rede era mensurada
através de observações, questionários e entrevistas \citep{Wasserman}.
Diferentemente de outras ciências sociais que consideravam apenas os indivíduo e
seus atributos, o estudo das redes sociais considera suas relações e os atributos
dessas relações. A rede social é um fenômeno complexo envolvendo os
relacionamentos de diversos atores em suas particularidades e que, através de um
processo que chamamos de mensuração, pode ser traduzido em uma representação.
Toda representação da rede social, por ser um modelo, é naturalmente parcial e
enviesado. Comumente, as pesquisas de redes sociais trabalham com grafos onde os
vértices são os atores e os arcos entre os vértices são as relações mensuradas; e
matrizes, onde as linhas e colunas são os atores e a posição $(i,j)$ da matriz
representa o arco \textbf{do} ator $i$ \textbf{para} o ator $j$. Para economizar
repetições, no decorrer deste trabalho quando estivermos nos referindo ao
fenômeno observado, utilizaremos o termo \textbf{rede social observada}, enquanto
que os termos \textbf{representação} e \textbf{rede social} serão
intercambiáveis.

Devido à disponibilidade de ferramentas matemáticas para o tratamento de grafos,
a análise de redes sociais desenvolveu-se rapidamente construindo métodos e
modelos estatísticos apropriados \citep{Butts2009}. A partir desse ferramental,
o ramo das ciências sociais passou a quantificar diversos fenômenos antes
considerados apenas do ponto de vista subjetivo, como a proeminência dos atores,
que estaria relacionada com a sua centralidade no grafo.

\section{Redes sociais digitais}
\label{sec:redes_dig}
Com a popularização da Internet é fato que pessoas se conectam umas às outras
virtualmente por seu intermédio. Os mecanismos de interação à disposição vão da
simples troca de mensagens, à venda e troca de produtos, à participação conjunta
em jogos \textit{multiplayer} massivos. Indo além do que sociólogo algum sonhou
realizar no início dos estudos de redes sociais, grande parte dessas interações
estão registradas, ou podem ser registradas eletronicamente a baixo custo,
fornecendo uma quantidade nunca antes disponível de informações para
estudos antropológicos e sociais da rede.

E assim tem sido, desde o nível micro com a análise dos conteúdos trocados entre
as interações pontuais de alguns indivíduos \citep{Recuero2008}, passando por
análise de potencial de marketing \citep{Clemons2007, Domingos2001,
Richardson2002, Ma2008}, busca de pessoas \citep{ADAMIC2005}, de especialistas
\citep{Ehrlich2007}, formação de grupos \citep{Adamic2003, Backstrom2006,
Kumar2006}, divulgação de notícias \citep{Gruhl2004}, dinâmicas de prestígio
\citep{Salganik2006, Song2007}.

Enquanto nosso objetivo é alcançar resultados similares as pesquisas anteriores
de influência em redes sociais digitais, decidimos antes colocar a questão: como
mensurar a rede social digital? Cada pesquisa teve seu critério: quantidade de
e-mails trocados, recomendações, similaridades de perfil, participação nas mesmas
comunidades. Dissemos no começo que a rede social observada é um fenômeno que
pode ser representado, mas que não é a representação em si, por esta razão, toda
representação possui um viés. Ora, ao acrescentarmos digital ao termo,
queremos dizer que estamos tratando da observação do fenômeno através de mídias
digitais; não mais das interações ao vivo e analógicas, mas através de
ferramentas eletrônicas que permitem a fácil armazenação, indexação e recuperação
dessa informação.

Para responder essa questão precisamos definir quais ferramentas são essas. Uma
resposta óbvia seria sites de relacionamento (ou sites de redes sociais),
definido como sendo um espaço (virtual) em que seja possível 1) criar um perfil,
2) relacionar uma lista públicade amigos, 3) navegar por essa rede de perfis
interligados \citep{Boyd2007}. Porém tais sites são apenas um dentre muitos tipos
de ferramentas que podem ser analisados, como por exemplo: fóruns, listas de
discussão, sites de compartilhamento de conteúdo, comércio eletrônico,
\emph{blogs}, \emph{microblogs} (e.g., Twitter), salas de bate-papo. Por questões
de privacidade deixaremos de lado as formas pessoais de interação, como
\emph{instant messengers} e e-mails.

Ou seja, qualquer espaço (virtual) em que se é possível 1) identificar
unicamente um ator, 2) mapear atores agentes e receptores a uma determinada
interação com suas propriedades, pode ser insumo para a mensuração da rede. Mais
adiante veremos que idealmente também será necessário demarcar a posição dessa
interação no tempo, para possibilitar uma análise longitudinal da evolução da
rede. Chamamos de \textbf{medianeiro} qualquer espaço (virtual) que satisfaça a
condição acima.

Porque nos parece evidente a impossibilidade de aplicar questionários ou
entrevistas com centenas de milhares de atores, respondemos a questão de como
mensurar a rede assim: através das interações encontradas nos medianeiros. A
mensuração é um processo de mineração de dados e, portanto, sujeita a todos os
empecilhos típicos do campo como informações incompletas, ruidosas, esparsas,
redundantes. A questão que nos aparece agora é como as interações observadas
combinam-se para formar tal rede e se ela é significativa para a análise de
proeminência.

\section{Mensurando a rede}
\label{ch:kddm}

\emph{Knowledge discovery} é o processo ``... não-trivial de identificar,
em dados, padrões válidos, novos, potencialmente úteis e ultimamente
compreensíveis'' \citep{Fayyad1996} e mineração de dados é uma de suas etapas.
Chamamos de mensuração da rede social digital o processo de minerar dados
proveninentes de meios digitais de interação social para formar uma
representação. Na última década alguns métodos norteadores para os projetos de
mineração de dados foram propostos, dentre os quais escolhemos o método de 6
passos descritos por \cite{Cios2005} e que consiste em sua forma geral:

\begin{description}
\item[1. Entendendo o domínio do problema]Neste passo devemos determinar os
objetivos do projeto e aprender sobre as possíveis técnicas conhecidas
para alcançá-los.
\item[2. Entendendo os dados]Este passo inclui coletar os dados, decidir quais
serão utilizados, priorizar atributos, verificar sua utilidade em relação aos
objetivos. Os dados precisam ser verificados em termos de completude,
plausabilidade, etc.
\item[3. Preparação dos dados]Este é o passo chave do qual o sucesso de todo o
processo depende; Ele geralmente consome metade de todo o esforço da mineração.
Aqui, decidimos quais dados serão usados como entradas para quais técnicas de
mineração do passo 4. O que pode envolver levantar amostragem de dados, executar
testes de correlação e significância, remoção e correção de ruído, etc. Os dados
tratados, depois poderão ser processados para a seleção de características,
redução da dimensionalidade, derivação de novos atributos (discretização) e
agregação dos dados (granularização). O resultado é um novo conjunto de dados
que atendem a requesitos específicos, necessários para sua utilização como
entrada para as ferramentas de mineração.
\item[4. Mineração dos dados]Aplicação dos métodos de mineração selecionados.
Apesar de ser através das ferramentas de mineração que as novas informações
são descobertas, sua utilização normalmente envolve menos esforço do que
preparar os dados. Ferramentas de mineração reunem diversos tipos de algoritmos
como conjuntos \emph{fuzzy}, métodos Bayesianos, computação genética,
aprendizado de máquina, redes neurais, etc. Para uma visão mais detalhada desses
algoritmos, referimo-nos a \cite{JiaweiHan2006}.
\item[5. Avaliação do conhecimento descoberto]Interpretação dos
resultados, verificando a relevância da informação encontrada. Somente os
modelos aprovados são mantidos, todo o processo pode ser revisitado e ações
alternativas que levem à melhoria dos resultados podem ser identificadas.
\item[6. Usando o conhecimento descoberto]Entrega do conhecimento produzido.
Criação de um plano para monitorar sua utilização, documentação do projeto,
estender sua aplicação para outros domínios.
\end{description}

O método de 6 passos para mineração de dados foi escolhido dentre outros
possíveis devido ao seu viés acadêmico e modelo iterativo com ciclos de
\emph{feedback} explícitos que orientam o retorno a passos anteriores para a
melhoria do processo \citep{KURGAN2006}. Mostraremos agora, para cada passo do
método quais as considerações necessárias, dificuldades e possíveis soluções no
contexto da mensuração de redes sociais digitais para a análise da influência.

\section{Contribuição e metodologia}

Uma vez definidos o conceito de rede social digital e sua mineração nas seções
anteriores, este trabalho aprofunda cada um dos passos da metodologia de
\citeauthor{Cios2005} a partir do ponto de vista da análise de influência. No
\chapref{ch:dominio} revisitamos a teoria de força da conexão de
\citeauthor{Granovetter1973} para entedermos sua consequência nas abordagens
atuais de mensuração de redes sociais digitais. No \chapref{ch:dados}
apresentamos uma tipologia de interações que facilita o entendimento das
diferentes formas conexões entre dois indivíduos nos meios digitais e da mesma
forma a sua mensuração em uma representação da rede; também lançamos mão da
teoria da atenção de \citeauthor{Davenport2001} para tecer um algoritmo simples
que integre diferentes tipos de interações textuais em uma única representação.
O \chapref{ch:preparacao} traz detalhes de operação do algoritmo de mensuração
da atenção e também critérios para comparar diferentes representações da rede
para a escolhar do melhor subconjunto que será utilizado na análise de
influência. Sobre a análise de influência propriamente dita, trataremos no
\chapref{ch:mineracao}, onde também relacionaremos o conceito de comunidades de
prática com a avaliação da aplicabilidade dos resultados da análise. O
\chapref{ch:amigos} contém a descrição de um experimento realizado com uma rede
real, onde será avaliado a utilidade da metodologia de \citeauthor{Cios2005}
como processo orientador da mineração e também a utilidade do algoritmo de
mensuração da atenção proposto. Encerramos com a conclusão e trabalhos futuros.