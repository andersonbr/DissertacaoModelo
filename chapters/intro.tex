\chapter{Introdução}
\label{ch:introducao}
A análise de redes sociais tem sido utilizada para a investigação
de temas tão diversos quanto a difusão de inovações \citep{Coleman1966},
oportunidades de emprego \citep{Granovetter1995}, prevenção contra fraude
\citep{Neville2005} e marketing \citep{Domingos2001}. Muito dessa pesquisa
inicial se baseia em redes pequenas em torno de indivíduos escolhidos por amostragem
\citep{Wasserman}\citep{Newman2006}, porém a recente disponibilidade de
informações relacionais na internet através de sites de relacionamentos permitiu o
desenvolvimento da análise de redes em larga escala \citep{Boyd2007}. Não
obstante, o meio ainda carece de modelos dinâmicos de representação para redes
sociais observadas a partir do fenômeno digital \citep{Xiang2010} e é justamente
nesse ponto que o trabalho atual se concentra. Nosso objetivo é avaliar as
dificuldades, parâmetros e modelos existentes para a representação e modelagem
não-supervisionada de redes sociais digitais em larga escala e tempo real,
especificamente para aplicações que façam uso da rede para identificar atores
chaves em processos de difusão de conhecimento, inovações e recursos.

No \chapref{ch:redes} introduziremos os conceitos de redes sociais, redes sociais
digitais e mensuração. Depois aprofundaremos o conceito de mensuração da rede a
análise de influência trazendo métodos da mineração de dados e \emph{knowledge
discovery} no \chapref{ch:kddm}. A partir daí nos Capítulos \ref{ch:dominio},
\ref{ch:dados}, \ref{ch:preparacao} e \ref{ch:mineracao} visitamos cada passo do
processo delineando as dificuldades e soluções existentes no tratamento de dados
relacionais, na agregação das interações em uma representação da rede e no
análise de influência através de técnicas de cálculo de proeminência
tradicionais; também introduziremos uma forma de agregar as interações através do
montante de atenção investido. Finalmente encerramos com a conclusão e trabalhos
futuros.

Escolhemos separar seções técnicas do fluxo principal do texto, assim um leitor
interessado em compreender as idéias aqui descritas pode evitar facilmente
seções que não sejam do seu interesse. Um $\bigotimes$ no título da seção indica
que seu conteúdo tem alguma matemática envolvida.
