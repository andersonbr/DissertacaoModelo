\chapter{Introdução}
\label{ch:introducao}
A análise de redes sociais tem sido utilizada para a investigação de problemas
tão diversos quanto a difusão de inovações \citep{Coleman1966}, oportunidades de
emprego \citep{Granovetter1995}, prevenção contra fraude \citep{Neville2005} e
marketing \citep{Domingos2001}. Muito dessa pesquisa inicial se baseia em redes
pequenas em torno de indivíduos escolhidos por amostragem
\citep{Wasserman}\citep{Newman2006}, porém a recente disponibilidade de
informações sobre as conexões entre indíviduos através de sites de
relacionamentos na internet permitiu o desenvolvimento da análise de redes em
larga escala \citep{Boyd2007}. Não obstante, ainda há carência de modelos
dinâmicos para representar redes sociais observadas a partir do fenômeno
digital \citep{Xiang2010} e é justamente nesse ponto que o trabalho atual se
concentra. Nosso objetivo é avaliar as dificuldades, parâmetros e modelos
existentes para a representação e modelagem não-supervisionada de redes sociais
digitais em larga escala e tempo real, especificamente para aplicações que façam
uso da rede para identificar atores chaves em processos de difusão de
conhecimento, inovações e recursos.

\section{Marketing viral}
\label{sec:viral}

Em 1996, Sabeer Bhatia e Jack Smith fundam um serviço de e-mail baseado
puramente em HTML chamado de Hotmail. Um ano e meio depois, o Hotmail já contava
com uma base de 12 milhões de usuários, sendo vendido para a Microsoft por US\$
400 milhões. Sua concorrente mais próxima, o serviço de e-mail Juno, levara o
dobro do tempo para conseguir um terço dessa quantidade de usuários, 4 milhões,
gastando em torno de US\$ 20 milhões em publicidade. O Hotmail utilizou menos de
US\$ 500 mil em propaganda. Esse é o caso clássico de \emph{Marketing Viral}.

O Hotmail e muitos outros serviços na internet depois dele utilizam-se do boca a
boca de seus usuários para se promover. O potencial de \emph{marketing} de um
indivíduo depende do seu entorno na rede social, isto é, a quem ele está
conectado através de interações e de sua capacidade de influenciá-los. Aqueles
com grande potencial são os chamados atores-chaves e através da análise de
influência é possível mapeá-los. Uma vez identificados, os atores-chaves podem
ser envolvidos em diferentes processos de difusão pela rede atuando como fontes
de informação e provocando um efeito cascata em maior escala que aquele
provocado pela escolha aleatória das fontes.

@*&(#!*&(#@!&(* AGORA REDES)))

No \chapref{ch:redes} introduziremos os conceitos de redes sociais, redes sociais
digitais e mensuração. Depois aprofundaremos o conceito de mensuração da rede a
análise de influência trazendo métodos da mineração de dados e \emph{knowledge
discovery} no \chapref{ch:kddm}. Nos Capítulos \ref{ch:dominio}, \ref{ch:dados},
\ref{ch:preparacao} e \ref{ch:mineracao} visitamos cada passo do processo
delineando as dificuldades e soluções existentes no tratamento de dados
relacionais, na agregação das interações em uma representação da rede e na
análise de influência através de técnicas de cálculo de proeminência
tradicionais; também introduziremos uma forma de agregar as interações através do
montante de atenção investido. Finalmente encerramos com a conclusão e trabalhos
futuros.