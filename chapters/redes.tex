\chapter{Redes Sociais}
\label{ch:redes}
O estudo de rede sociais inicia nas décadas de 40 e 50, inicialmente voltado para
o estudo de pequenos grupos de individuos e suas interações, a rede era mensurada
através de observações, questionários e entrevistas \citep{Wasserman}.
Diferentemente de outras ciências sociais que consideravam apenas os indivíduo e
seus atributos, o estudo das redes sociais considera suas relações e os atributos
dessas relações. A rede social é um fenômeno complexo envolvendo os
relacionamentos de diversos atores em suas particularidades e que, através de um
processo que chamamos de mensuração, pode ser traduzido em uma representação.
Toda representação da rede social, por ser um modelo, é naturalmente parcial e
enviesado. Comumente, as pesquisas de redes sociais trabalham com grafos onde os
vértices são os atores e os arcos entre os vértices são as relações mensuradas; e
matrizes, onde as linhas e colunas são os atores e a posição $(i,j)$ da matriz
representa o arco \textbf{do} ator $i$ \textbf{para} o ator $j$. Para economizar
repetições, no decorrer deste trabalho quando estivermos nos referindo ao
fenômeno observado, utilizaremos o termo \textbf{rede social observada}, enquanto
que os termos \textbf{representação} e \textbf{rede social} serão
intercambiáveis.

Devido à disponibilidade de ferramentas matemáticas para o tratamento de grafos,
a análise de redes sociais desenvolveu-se rapidamente construindo métodos e
modelos estatísticos apropriados \citep{Butts2009}. A partir desse ferramental,
o ramo das ciências sociais passou a quantificar diversos fenômenos antes
considerados apenas do ponto de vista subjetivo, como a proeminência dos atores,
que estaria relacionada com a sua centralidade no grafo.

\section{Alguma Formalização}

Representamos a rede como um grafo $G(V,E)$ onde $V$ é um conjunto de $n$
vértices (também chamados de atores) e $E$ o conjunto de arestas que os conectam.
Quando a representação é não-direcionada o par $(i,j)$ é chamado de aresta e
temos que $(i,j) \in E$ implica $(j,i) \in E$. No caso em que há direcionamento,
essa afirmação não se sustenta e por isso $(i,j) \in E$ não implica em $(j,i) \in
E$ necessariamente, além do que chamamos o par de arco de $i$ para $j$. De
maneira geral, sempre trataremos a representação como sendo direcionada.

Definimos a matriz de adjacência da rede $X$ com $n$ linhas e colunas, onde
$x_{ij} = 1$, se $(i,j) \in E$ ou 0 de outra forma. Essa matriz de $X$ é uma
representação binária da rede. Podemos generalizar essa notação para incluir
redes com valores discretos onde $0 \leq x_{ij} \leq m$ ou contínuos dispostos em
um intervalo. Ambas as notações, de grafo e de matriz, serão usadas de forma
intercalada neste trabalho.

\section{Redes sociais digitais}
\label{sec:redes_dig}
Com a popularização da Internet é fato que pessoas se conectam umas às outras
virtualmente por seu intermédio. Os mecanismos de interação à disposição vão da
simples troca de mensagens, à venda e troca de produtos, à participação conjunta
em jogos \textit{multiplayer} massivos. Indo além do que sociólogo algum sonhou
realizar no início dos estudos de redes sociais, grande parte dessas interações
estão registradas, ou podem ser registradas eletronicamente a baixo custo,
fornecendo uma quantidade nunca antes disponível de informações para
estudos antropológicos e sociais da rede.

E assim tem sido, desde o nível micro com a análise dos conteúdos trocados entre
as interações pontuais de alguns indivíduos \citep{Recuero2008}, passando por
análise de potencial de marketing \citep{Clemons2007, Domingos2001,
Richardson2002, Ma2008}, busca de pessoas \citep{ADAMIC2005}, de especialistas
\citep{Ehrlich2007}, formação de grupos \citep{Adamic2003, Backstrom2006,
Kumar2006}, divulgação de notícias \citep{Gruhl2004}, dinâmicas de prestígio
\citep{Salganik2006, Song2007}.

Enquanto nosso objetivo é alcançar resultados similares as pesquisas anteriores
de influência em redes sociais digitais, decidimos antes colocar a questão: como
mensurar a rede social digital? Cada pesquisa teve seu critério: quantidade de
e-mails trocados, recomendações, similaridades de perfil, participação nas mesmas
comunidades. Dissemos no começo que a rede social observada é um fenômeno que
pode ser representado, mas que não é a representação em si, por esta razão, toda
representação possui um viés. Ora, ao acrescentarmos digital ao termo,
queremos dizer que estamos tratando da observação do fenômeno através de mídias
digitais; não mais das interações ao vivo e analógicas, mas através de
ferramentas eletrônicas que permitem a fácil armazenação, indexação e recuperação
dessa informação.

Para responder essa questão precisamos definir quais ferramentas são essas. Uma
resposta óbvia seria sites de relacionamento (ou sites de redes sociais),
definido como sendo um espaço (virtual) em que seja possível 1) criar um perfil,
2) relacionar uma lista públicade amigos, 3) navegar por essa rede de perfis
interligados \citep{Boyd2007}. Porém tais sites são apenas um dentre muitos tipos
de ferramentas que podem ser analisados, como por exemplo: fóruns, listas de
discussão, sites de compartilhamento de conteúdo, comércio eletrônico,
\emph{blogs}, \emph{microblogs} (e.g., Twitter), salas de bate-papo. Por questões
de privacidade deixaremos de lado as formas pessoais de interação, como
\emph{instant messengers} e e-mails.

Ou seja, qualquer espaço (virtual) em que se é possível 1) identificar
unicamente um ator, 2) mapear atores agentes e receptores a uma determinada
interação com suas propriedades, pode ser insumo para a mensuração da rede. Mais
adiante veremos que idealmente também será necessário demarcar a posição dessa
interação no tempo, para possibilitar uma análise longitudinal da evolução da
rede. Chamamos de \textbf{medianeiro} qualquer espaço (virtual) que satisfaça a
condição acima.

Porque nos parece evidente a impossibilidade de aplicar questionários ou
entrevistas com centenas de milhares de atores, respondemos a questão de como
mensurar a rede assim: através das interações encontradas nos medianeiros. A
mensuração é um processo de mineração de dados e, portanto, sujeita a todos os
empecilhos típicos do campo como informações incompletas, ruidosas, esparsas,
redundantes. A questão que nos aparece agora é como as interações observadas
combinam-se para formar tal rede e se ela é significativa para a análise de
proeminência.