\chapter{Entendendo os dados}
\label{ch:dados}

A primeira crítica que deve ser avaliada é a de que interações digitais não são
um indicador confiável da relação entre dois indivíduos \citep{Clemons2007}. Que
se um amigo virtual não é mais do que um conhecido, não há relação de influência
significativa entre eles. Porém, as pesquisas em redes sociais digitais tem
demonstrado que em sua maior parte, os indivíduos utilizam os meios digitais para
continuar uma relação já existente no mundo \emph{off-line}
\citep{Haythornthwaite2005, Recuero2008, Sassen2002}. Outra vantagem
de ater-se a interação observada é que ela não carrega alguns pontos fracos da
abordagem direta (questionário, entrevista) que é o esquecimento e a omissão das
relações nas respostas (a taxa de erro quando interrogados sobre as interações
que mantiveram chega a 50\% em comparação com a observação) \citep{Mislove2007}.
Mas a vantagem mais óbvia e determinante é o custo, coletar e processar os dados
advindo das interações dos atores no meio digital é muito mais barato que fazer o
mesmo para interações não digitais, na medida em que o tamanho da rede aumenta,
ou mesmo em comparação com entrevistas e questionários. Na casa dos milhões de
membros na rede, qualquer coisa que não seja a primeira opção é atualmente
inviável.

Nessa etapa da mineração, uma vez delineada as técnicas de mensuração, é
necessário escolher quais medianeiros serão utilizados. Recapitulando a definição
de medianeiro: é todo espaço (virtual) em que seja possível 1) definir unicamente
um ator, 2) mapear atores agentes e recepetores a uma interação e suas
propriedades. Alguns exemplos são: salas de batepapo, fórums, lista de discussão,
sites de relacionamentos, sites de compartilhamento de fotos e vídeos. Para cada
medianeiro, o pesquisador pode ter maior ou menor acesso à informação.

Quando a única informação disponível são as publicadas nas páginas da internet
ou outros meios de acesso ao medianeiro, dizemos que sua análise é
\textbf{extrínseca}. A grande maioria dos trabalhos em redes sociais digitais
hoje é feito dessa forma com a confecção de \emph{crawlers}, programas de
computador que percorrem conteúdos \emph{on-line} retirando informações
estruturadas de dados semi-estruturados próprios para o uso humano, como as
\emph{web pages}. Esse tipo de análise é limitada muitas vezes à interação
presumida, já que geralmente através dela é possível saber que um ator publicou
conteúdo na comunidade, mas não quem da comunidade parou para vê-lo.

Quando as informações são coletadas diretamente dos banco de dados do
medianeiro, chamamos essa análise de \textbf{intrínseca}. A análise intrínseca
encerra diversas vantagens, principalmente por ter acesso ao funcionamento da
rede, podendo coletar dados sobre seu uso. São informações como mensagem lidas e
não lidas, tempo usado em cada uma, e que formam um tipo de interação que
chamaremos de \textbf{passiva}.

Porém muitos são os tipos de interação digitais encontradas e enquanto não
tivermos um maior entendimento das suas semelhanças e diferenças, não nos será
possível construir uma abordagem integrada de mensuração. Por uma tipologia das
interações digitais é então que nossa atenção deve agora se voltar.

\section{Por uma tipologia das interações digitais}
\label{sec:tipologia}
O estudo das interações humanas é o objeto das ciências sociais, notadamente, no
contexto micro, da antropologia e etnografia. Somente com a popularização da
internet é que as interações digitais (mediadas por computador) ocuparam maior
destaque nesse meio \citep{Wellman1996, Herring2002}. Uma tipologia inicial
foi proposta por \citet{Burnett2000} e revisada por \citet{Burnett2004}, porém o
seu foco é na distinção entre interações direcionadas e as não-direcionadas para
a aquisição de informação. Nossa tipologia começa dela, mas expande incorporando
conceitos de capital social \citep{Recuero2008} e abrangência
\citep{MARTINEZ2000}.

Podemos as interações digitais dividir por tipo do \textbf{conteúdo},
\textbf{abrangência} e \textbf{intenção}. Os tipos de conteúdos podem ser
divididos em dois grandes grupos: textuais e não textuais. Abrangência consiste
se a interação é individual básica, individual desenvolvida, individual
generalizada ou comum. Finalmente, quanto a intenção podemos classificar a
interação em afirmativa, negativa, conversacional, informativa, conectiva. É
importante ter em mente que as categorias apresentadas não são, de forma alguma,
mutuamente exclusivas, podendo mesmo numa interação singular ser combinadas em
diferentes formas.

\subsection{Por conteúdo}

O tipo do conteúdo da comunicação diz muito sobre o capital social que carrega
\citep{Kim2007}, por exemplo, mensagens síncronas possuem vocabulário mais
limitado, são mais informais e possuem maior carga social fática
\citep{Danet1998, Ko1996, WERRY1996} enquanto que mensagens assíncronas tendem a
ser maiores, mais multifuncionais e linguisticamente complexas
\citep{Herring1999}. De acordo com essas diferenças, mensagens síncronas parecem
ser mais apropriadas para a interação social enquanto que mensagens assíncronas o
são para discussões mais complexas e resolução de problemas.

Também não é nosso objetivo ainda nos aprofundarmos numa análise do discurso
mediado por computador \citep{Herring2001}. A classificação que utilizaremos 
aqui usa o conceito de protótipo e por tanto tem expressividade reduzida diante
do surgimento de formas interação inovadoras \citep{Herring2007}. Porém ela será
suficiente nesse estudo inicial sobre mensuração de redes para a combinação de
diferentes interações. Dividimos inicialmente em duas grandes categorias:
textual e não textual; para cada uma então são listados seus principais
protótipos.

\begin{description}
\item[conteúdo textual] Já foi dito o quanto o texto é importante para a
comunicação mediada por computador, mesmo numa época em que o compartilhamento
de vídeos está na moda, o texto, na forma de comentários, continua sendo o
principal móvel das trocas sociais \citep{Herring2002}. São desse tipo:
\begin{itemize}
  \item comentários;
  \item mensagens;
  \item tópicos;
  \item \emph{blogs} e similares;
  \item \emph{microblogging} e similares;
  \item descrições.
\end{itemize}
\item[conteúdo não-textual] Nesse grupo estão relacionados não só as interações
áudiovisuais, mas também as interações ``mudas'' como a formação de laços
explícito entre os atores, a classificação mútua dicotômica ou graduada e a
recomendação de conteúdos de terceiros. Exemplos:
\begin{itemize}
  \item fotos;
  \item vídeos;
  \item \emph{links};
  \item laços explícitos entre os atores;
  \item classificação (\emph{rating}, \emph{ranking}, favoritos);
  \item compartilhamento.
\end{itemize}
\end{description}

\subsection{Por abrangência}

A abrangência define quais são os atores influenciados pela interação, em
outras palavras, os participantes do ``discurso" \citep{Dooley2001}. É através da
abrangência da interação que o processo de mensuração recupera os atores
participantes na conexão mensurada e por tanto representa a forma como os
agentes buscam se posicionar na rede como um todo. Essa classificação foi
apresentada uma primeira vez em \citep{MARTINEZ2000}.
\begin{description}
\item[individual básica] Classificam-se neste grupo as interações de caráter
privativo que partem de um ator específico para outro ator. As interações que
tipicamente pertencem a este grupo são as mensagens, os pedidos de conexão e o
compartilhamento de conteúdo do tipo ``\textit{forward}''.
\item[individual desenvolvida] Quando a interação se inicia em um ator e envolve
sua rede imediata de contatos sem estar publicamente disponível para qualquer
membro da rede. São desse grupo, em sua maioria: tópicos de fóruns, fotos,
vídeos, compartilhamentos, microblogging.
\item[individual generalizada] Quando a interação se inicia em um ator e
torna-se pública para todos os atores da rede. Todos os tipos de interação podem
se classificar nesse grupo, depende do grau de livre acesso que a rede
proporciona aos seus membros.
\item[comum] Quando a interação se dá num espaço de igualdade, quer dizer, que
todos podem interagir com todos no mesmo nível, chamamos de interação comum. Um
exemplo claro dessa categoria é as salas de batepapo onde todos podem publicar
mensagens para todos lerem. Listas de discussão também seguem esse modelo.
Um \textit{Blog} em particular não é uma interação comum, na maioria dos casos,
porque só o mantenedor do \textit{blog} pode publicar artigos nele, já o
espaço de comentários do artigo possa ser considerado um espaço comum.
\end{description}

\subsection{Por intenção}

Por último, mas não por menos, temos a intenção sobre a qual o ator reveste sua
interação. A diferença de intenção não só pode representar mudança significativa
na força da influência, como necessariamente define os possíveis resultados da
interação. Em \citet{Recuero2008} encontramos uma classificação da intenção
das interações, que para o escopo proposto por esse trabalho é suficiente e as
divide em cinco grandes grupos:
\begin{description}
\item[afirmativa] Trata da afirmação de suporte social entre um ator e outro.
Pode ser um comentário positivo relacionado a uma interação prévia do ator
elogiado, uma avaliação positiva, a recomendação do seu conteúdo para outros.
\item[negativa] Quando a interação se dá para depreciar o outro. Pode ser por
comentário, tópicos, mensagens, avaliação negativa.
\item[conversacional] Quando a interação tem caráter pessoal, relacionado a uma
conversação que se inicia ou que está em andamento.
\item[informativo] Quando a intenção é informar um grupo de atores sobre
determinado assunto. Avisos, artigos, propaganda, críticas.
\item[conectivos] Quando a intenção é formar laços explícitos através da rede.
Pedidos para conexão como adicionar à lista de contatos/amigos/etc.
\end{description}
Das três dimensões de classificação da interação, a intenção é certamente a mais
difícil de aferir computacionalmente. Isso porque sua característica subjetiva o
que a torna também objeto ideal para a pesquisa qualitativa. Porém recente
evolução da mineração de opinião e análise de sentimentos pode trazer opções
para a análise de intenção no contexto da mensuração das redes sociais digitais.
Para uma compilação dos principais avanços e desafios na área, nos referimos a
\citet{Wilson2005, Ding2007, Pang2008}. Por sua importância e
dificuldade, teremos o cuidado de anotar mensurações de redes sociais digitais
que não levem em consideração a intenção, como \textbf{análises ingênuas}.

\subsection{Interações passivas}

Para finalizar essa seção sobre interações, falaremos aqui de interações passivas
e comportamentais que podem levar a um maior entendimento de como o ator investe
sua atenção na rede e, por isso mesmo, como se posiciona na rede em relação a
outros atores. Podemos considerar como interação passiva como aquela que não é
necessariamente percebida pelos outros atores além do interagente, nesse tipo
encontram-se todas as interações do ator com o sistema como: mensagens lidas, não
lidas, tempo usado para cada mensagem, perfis visitados, tempo usado na leitura
de outros conteúdos. Essas informações poderiam ser utilizadas numa análise
intrínseca da rede para a concepção de um modelo mais real do interagente quanto
ao dispêndio da atenção.

\subsection{$\bigotimes$ Notação}

Seja $\mathscr{X}$ um medianeiro, existirá um conjunto $\mathscr{L}$ de
interações associado a $\mathscr{X}$. Considere o conjunto $\mathscr{N}$ de
atores envolvidos através do medianeiro $\mathscr{X}$. Temos que para cada
iteração $l_k \in \mathscr{L}$ existe pelo menos um ator $n_i \in \mathscr{N}$
que provocou a interação, chamamos esses atores de agentes ou autores. Também
existe pelos menos um ator $n_j \in \mathscr{N}$ que recebeu a interação,
chamamos esses atores de receptores, audiência ou abrangência. Assim para fins de
notação, sendo $\mathscr{P}(.)$ o conjunto das partes, considere
$A:\mathscr{L}\to\mathscr{P}(\mathscr{N})$ onde $A(l_k)$ é conjunto de autores de
$l_k$. Da mesma forma, faça $R:\mathscr{L}\to\mathscr{P}(\mathscr{N})$ onde
$R(l_k)$ é conjunto de receptores de $l_k$. Nenhuma das duas funções são
inversíveis, porém para economizar a notação definimos
$A^{-1}:\mathscr{N}\to\mathscr{P}(\mathscr{L})$ onde $A^{-1}(n_i)$ é o conjunto
de interações $l$ tal que $n_i \in A(l)$. O mesmo deve ser considerado para
$R^{-1}$.

\section{Uma teoria da atenção como capital social}
\label{sec:teoria_atencao}

Antes de prosseguir para a etapa seguinte na mensuração da rede, gostaríamos de
discutir uma aproximação para a influência entre os atores: atenção como
capital social. Capital social é todo recurso que mantém a rede social
\citep{Coleman1988} e pode ser mobilizado através das conexões
\citep{Gyarmati2004}. Exemplos de capital social são a confiança e o suporte
emocional. Por esta razão uma outra forma de chamar as dimensões de força da
conexão de Granovetter é de capital social, i.e., quanto maior o fluxo de
capital social que a conexão suporte, maior é a sua força. Mostraremos que a
atenção não só atende a definição de capital social, como se aproxima mais do
conceito de influência e, certamente, é mais fácil de medir.

Foi o vencedor do prémio nobel, o economista Herbert Simon, que disse:

\begin{quote}{\citep{Simon1996}}\emph{What information consumes is rather
obvious: it consumes the attention of its recipients. Hence a wealth of information
creates a poverty of attention }
\end{quote}

Vivemos em uma era de riqueza de informação e por esta razão estamos em escassez
de atenção \citep{Goldhaber1997}. Atenção é vista atualmente como \textbf{o} mais
importante recursos para as organizações \citep{Davenport2001}. Ela também assume
papel crucial na formação e manutenção de relacionamentos e como é um recurso
limitado, o mais comum é que cada um invista nas relações que percebam ser mais
importantes \citep{Dindia1993}. Por esta razão a atenção satisfaz os critérios de
capital social, a saber 1) contribuir na manutenção da rede, 2) pode ser
mobilizada através das conexões. Vários estudos consideram a atenção como a nova
economia na internet, através de comentários, classificações positivas e
\textit{profiles} \citep{Humphreys2009, Wu2009, Skageby2009}. Nessa economia nós
temos ``celebridades'' e ``fãs'', muito similar ao que encontramos no contexto da
influência.

Na verdade somente o fato de prestar atenção em si já é uma influência per si,
independente das futuras escolhas do receptor. Em uma pesquisa em Parma, na
Itália, por volta de 1991, neurocientistas conectaram cabos ao cérebro de macacos
de forma que o computador reconhecia pelo padrão de ativação dos neurônios quando
ele levava um amendoin à boca. Não obstante a ativação fosse registrada
eletrônicamente, eles também ligaram o aparato a um auto-falante de forma que
mesmo estando distantes soubessem quando o evento acontecia. Quando um aluno
passou pelo laboratório e levou uma banana à boca, o alarme soou. Porém o macaco
não estava fazendo nada além de observar o estudante. Estava claro que a atenção
no gesto do estudante disparava o cérebro a reproduzir (em sua mente) o movimento
como se fosse seu (\citealt{Rizzolatti1996}; apud \citealt{Goldhaber2006}).

Resta-nos saber como mediremos a atenção. Ora, em qualquer interação o que é
trocado em primeiro lugar entre o agente e os receptores é atenção. A atenção
flui não só do receptor para o agente ao receber a ``mensagem'', no que chamamos
de atenção \textbf{direta}, mas também do agente para os receptores, por a ter
preparado e comunicado, que chamamos de atenção \textbf{residual}
\citep{Goldhaber1997}. Apesar de que em quantias bem menores proporcionalmente,
por que o que cada receptor recebe é uma atenção ilusória, é uma fração da
atenção do agente e que o satisfaz de alguma forma na interação de modo que para
ele parece um bom negócio continuar retribuindo-a com a sua. Mas como essa
atenção se relaciona com as dimensões da força de Granovetter? Para responder
essa questão precisaremos usar nossa tipologia da interação.

Em primeiro lugar, frequência e duração são dimensões temporais que são
recuperadas a partir do tratamento longitudinal da rede. A frequencia das
interações e data do começo delas são métricas simples de colher e que contribuem
para força total. Quanto a intimidade, podemos verificar que a depender da
abrangência teremos espaço para maior ou menor intimidade, ou seja, quando mais
restrito a abrangência mais íntima a interação, sendo a mensagem de pessoa a
pessoa a forma mais íntima possível. Quanto a carga emocional, mesmo não tendo à
disposição meios de análise de sentimento, sabemos que as interações que mais nos
chamam atenção são as que possuem carga emocional \citep{Davenport2001}. Daí
podemos aproximar a dimensão de suporte emocional por uma combinação da de
duração, frequência e intensidade.

Para finalizar nossa consideração sobre a atenção como capital social, é
importante mencionar que a atenção não é direcionada apenas sobre um foco, mas
sobre o contexto. Isto quer dizer que alguns indivíduos podem receber atenção
indiretamente terem sido citados, por intermediarem a interação ou por
simplesmente fazer parte do contexto de alguma forma. Essa atenção chamamos de
\textbf{transitiva}.

\subsection{$\bigotimes$ Formalização}

Considere dois atores $i$ e $j \in \mathscr{N}$; se $\mathscr{X}$ é um medianeiro
para $\mathscr{N}$ com um conjunto de interações $\mathscr{L}$, considere também
$l$ uma interação de $\mathscr{L}$. Se $i \in A(l)$ e $j \in R(l)$ então existe
uma atenção direta $v_{ijl}\neq0$ de $i$ para $j$ e uma atenção residual de $j$
para $i$ expressa como $\dot{v}_{jil}\neq0$. Usaremos o símbolo $+$ em um índice
quando queremos dizer que trata-se do somatório sobre todo o seu conjunto, alguns
exemplos: $v_{i+l} = \sum_{j \in \mathscr{N}}v_ijl$ representa toda a atenção
direta investida por $i$ na interação $l$; $v_{ij+} = \sum_{l \in
\mathscr{L}}v_{ijl}$, toda atenção direta cedida de $i$ para $j$ considerando
todas as interações; e $v_{++l} = \sum_{i \in \mathscr{N}}\sum_{j \in
\mathscr{N}}v_{ijl}$, toda a atenção direta cedida na rede através da interação
$l$.

Algumas restrições são óbvias, como o total de atenção investida um ator qualquer
é: $v_{i++} + \dot{v}_{i++} = c, \forall i \in \mathscr{N}$. Essa restrição, no
entanto, não se sustenta quando só temos as interações da rede para avaliar já
que representam apenas uma fração da atenção total do ator.

Para definir a atenção transitiva é necessário primeiro estabelecer a relação
$T:\mathscr{L}\to\mathscr{L}$ tal que se $l$ e $q \in \mathscr{L}$ e uma faz
parte do contexto da outra, isto é, estão encadeadas, então $q \in T(l)$ se $q$
veio antes. Dizemos que há atenção transitiva $\widetriangle{v}_{ijl}$ se há um
caminho entre $i$ e $j$ através de $v$, isto é, se $i \in R(l)$ e $j \in A(q)$
para algum $q \in T(l)$. Exemplos de interações encadeadas são comentários de
\emph{posts}, \emph{threads} em fóruns e recomendação de conteúdo.
