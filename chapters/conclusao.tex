\chapter{Conclusão e trabalhos futuros}
\label{ch:conclusao}

Nas etapas finais do processo de descoberta do conhecimento, a avaliação e a
utilização, cada domínio fará a aplicação apropriada. Alguns critérios de
relevância para o caso do cálculo de proeminência foram discutidos na
\secref{sec:rel_proe}. Nosso intuito com esse trabalho foi reunir método e
técnicas para um processo pragmático de análise de influência de redes sociais
digitais, oferencendo um passo a passo compreensivo do que pode ser feito, das
dificuldades possíveis e das soluções existentes. Nesse sentido, sentimos que
esse \emph{framework} para a análise de influência em redes está ainda em sua
infância e há muito o que se fazer. Em tópicos gerais, reunimos abaixo os
principais trabalhos futuros:

\begin{itemize}
  \item Considerar a dimensão tempo na dinâmica da rede. Primeiro na
  evolução das conexões, seja através de séries temporais \citep{Snijders1996};
  como intervalos de tempo \citep{Butts2004}; ou como eventos pontuais no tempo
  \citep{Butts2008a}. Depois na evolução da afiliação dos atores a grupos
  através do tempo \citep{Berger-Wolf2006};
  \item Utilizar as informações baseadas em afiliações, estendendo
  modelos \emph{p*} para redes dois-modos (\emph{two-mode network})
  \citep{Field2006}. Em redes de dois-modos há mais de um tipo de nó, no nosso
  caso os atores seriam um tipo, as interações outro tipo. Essa visão
  hiperrelacional da interação permitiria diferenciar padrões de influência do
  tipo ``um para muitos'' dos do tipo ``muitos de um pra um'';
  \item Integrar com modelos estatísticos da economia da atenção para tratar
  fenômenos como a competição pela atenção, sobrecarga de informação e outras
  propriedades de mercado \citep{Falkinger2007}; 
  \item Aplicar o processo de mensuração descrito nesse trabalho a diferentes
  domínios, de forma a melhor avaliar as propriedades de seus parâmetros, sua
  utilidade na análise de influência e facilidade de aplicação.
\end{itemize}

Não obstante, acreditamos que a singular reunião da literatura espalhada sobre
o assunto em um \emph{framework} pragmático facilitará a construção de
ferramentas de mineração apropriadas para o problema. No \appref{ap:estudo} nos
dedicamos a desenvolver essa possibilidade.
