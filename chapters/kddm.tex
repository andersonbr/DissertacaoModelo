\chapter{Mensurando a rede}
\label{ch:kddm}

\emph{Knowledge discovery} é o processo maior ``... não-trivial de identificar,
em dados, padrões válidos, novos, potencialmente úteis e ultimamente
compreensíveis'' \citep{Fayyad1996} e mineração de dados é uma de suas etapas.
Chamamos de mensuração da rede social digital o processo de minerar dados
proveninentes de meios digitais de interação social para formar uma
representação. Na última década alguns métodos norteadores para os projetos de
mineração de dados foram propostos, dentre os quais escolhemos o método de 6
passos descritos por \cite{Cios2005} e que consiste em sua forma geral:

\begin{description}
\item[1. Entendendo o domínio do problema]Neste passo devemos determinar os
objetivos do projeto e aprender sobre as possíveis técnicas conhecidas
para alcançá-los.
\item[2. Entendendo os dados]Este passo inclui coletar os dados, decidir quais
serão utilizados, priorizar atributos, verificar sua utilidade em relação aos
objetivos. Os dados precisam ser verificados em termos de completude,
plausabilidade, etc.
\item[3. Preparação dos dados]Este é o passo chave do qual o sucesso de todo o
processo depende; Ele geralmente consome metade de todo o esforço da mineração.
Aqui, decidimos quais dados serão usados como entradas para quais técnicas de
mineração do passo 4. O que pode envolver levantar amostragem de dados, executar
testes de correlação e significância, remoção e correção de ruído, etc. Os dados
tratados, depois poderão ser processados para a seleção de características,
redução da dimensionalidade, derivação de novos atributos (discretização) e
agregação dos dados (granularização). O resultado é um novo conjunto de dados
que atendem a requesitos específicos, necessários para sua utilização como
entrada para as ferramentas de mineração.
\item[4. Mineração dos dados]Aplicação dos métodos de mineração selecionados.
Apesar de ser através das ferramentas de mineração que as novas informações
são descobertas, sua utilização normalmente envolve menos esforço do que
preparar os dados. Ferramentas de mineração reunem diversos tipos de algoritmos
como conjuntos \emph{fuzzy}, métodos Bayesianos, computação genética,
aprendizado de máquina, redes neurais, etc. Para uma visão mais detalhada desses
algoritmos, referimo-nos a \cite{JiaweiHan2006}.
\item[5. Avaliação do conhecimento descoberto]Interpretação dos
resultados, verificando a relevância da informação encontrada. Somente os
modelos aprovados são mantidos, todo o processo pode ser revisitado e ações
alternativas que levem à melhoria dos resultados podem ser identificadas.
\item[6. Usando o conhecimento descoberto]Entrega do conhecimento produzido.
Criação de um plano para monitorar sua utilização, documentação do projeto,
estender sua aplicação para outros domínios.
\end{description}

O método de 6 passos para mineração de dados foi escolhido dentre outros
possíveis devido ao seu viés acadêmico e modelo iterativo com ciclos de
\emph{feedback} explícitos que orientam o retorno a passos anteriores para a
melhoria do processo \citep{KURGAN2006}. Mostraremos agora, para cada passo do
método quais as considerações necessárias, dificuldades e possíveis soluções no
contexto da mensuração de redes sociais digitais para a análise da influência.