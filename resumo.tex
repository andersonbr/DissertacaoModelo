O valor de um membro em uma rede social é a influência que ele tem
sobre os processos da rede. Essa influência pode estar fragmentada em
múltiplas conexões periféricas ou em poucas conexões centrais, em
diferentes contextos ou em uma área de especialização, em toda a rede
ou em comunidades específicas. O reconhecimento de atores-chaves nos
processos de difusão de inovações em redes sociais pode levar a
campanhas de marketing viral mais eficientes, a identificação de
especialistas, a reconhecer fraquezas na segurança da rede e a
promover a adoção de conhecimentos em redes sociais de aprendizagem. O
presente estudo reune as abordagens utilizadas para a mensuração de
redes sociais digitais, aprensenta seus desafios e também suas
possíveis soluções dentro de uma metodologia e vocabulário até então
pouco explorados na literatura do assunto.

\begin{keywords}
social network, social influence, SNA, KDDM, tie strength,
relationship modeling, viral marketing, information diffusion;
\end{keywords}